\chapter{Conclusions and further research}\label{ch:conclusions}



In Chapter \ref{ch:Julia} a framework for deriving a pre-test design and for doing simulation studies on estimation of parameters of unidimensional 1PL and 2PL IRT models has been provided. It includes: first, the implementation of the cubic-spline method for interpolation and extrapolation of the masses on the starting quadrature points of the ability continuum; secondly, the definition of a bootstrap algorithm applied to IRT models for retrieving the empirical distribution function of the item parameters. All the described procedures are coded in \texttt{Julia}.

Regarding the first contribution, the results of the simulation study showed that our approach produces similar results in terms of estimation accuracy compared to the benchmark software, the \texttt{mirt} \texttt{R} package. Remarkable positive differences in favour of our algorithm are observed in case where the latent variable is not normally distributed. Overall, our algorithm shows always the best computational performances. The latter result demonstrates that \texttt{Julia} is a programming language suitable for numerical analysis, with a potential to be exploited by statisticians interested in optimization, of which likelihood maximization is a special case. 
The efforts we put in this direction are concretized with the production of a \texttt{Julia} package called \texttt{IRTCalibration} which, at the date of publication of this thesis is still under development and available under private request to the author. We decided to not make the code public yet because the documentation is still under development.

Nevertheless, we would like to point out that the \texttt{R} package \texttt{mirt} offers a multitude of options for latent  models such as multidimensional latent structure and different EM algorithms (such as stochastic EM and Monte Carlo EM). Also, their output is very rich; it produces standard errors and other model diagnostics. Our tool, instead, provides only the estimates of the latent variable, the final weights of the latent distribution, the calibrated IRT item parameter and their bootstrapped standard errors, since it is out of the scope of this work to inspect model diagnostics and other statistics. 

On the other hand, the second contribution allowed the analysis of the empirical distribution functions of the item parameters. Thus, it showed that the bootstrap is a powerful, prior free, tool to inspect the uncertainty of the item parameters because is able to capture the full characterization of the variability of the items due to the sample error. We believe that, in the future, a better specification of the parametric scheme would improve also the accuracy of the parameters estimation.

In Chapter \ref{ch:CC}, an ATA model which can deal with uncertainty in item parameters has been defined and a solver based on the simulated annealing meta-heuristic has been developed.
Again, the results of a simulation study showed that our solver outperformed the benchmark software \texttt{CPLEX} in several classical test assembly tasks. About the CC MAXIMIN ATA model we proved that our solver is able to look for the highest quantiles of the TIFs of the assembled tests successfully optimizing the specified models. The use of our model will allow the test assemblers to provide a conservative version of their tests which takes into account the effect of uncertainty of the item parameter estimates in the future process of ability estimation. The task the tests are primarily devoted to.
Moreover, a second package called \texttt{CCATA}, always based on the \texttt{Julia} programming framework, has been developed. Also in this case the code is available upon private request to the author. 

In the near future we intend to provide the documentation of the developed packages in order to make them available on a public repository (like github). It will follow an enrichment of the tools included within the \texttt{IRTCalibration} package, such as the possibility to input polytomous responses and estimate the parameters of multidimensional latent models. 
Regarding the test assembly suite \texttt{CCATA} we would like to implement other objective functions which until now couldn't be solved by linear ATA models.
Likely, the algortihms described in this thesis will be material for published scientific articles.
More in general, we believe that the heuristic we proposed can be applied to a wider class of optimization problems, outside of the ATA field, distinguished by having several binary optimization variables. It will be our future concern to extend its field of application in order to show if it can be helpful in other contexts.




